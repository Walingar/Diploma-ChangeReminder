\documentclass[times]{itmo-student-thesis}

%% Опции пакета:
%% - specification - если есть, генерируется задание, иначе не генерируется
%% - annotation - если есть, генерируется аннотация, иначе не генерируется
%% - times - делает все шрифтом Times New Roman, собирается с помощью xelatex
%% - languages={...} - устанавливает перечень используемых языков. По умолчанию это {english,russian}.
%%                     Последний из языков определяет текст основного документа.

%% Делает запятую в формулах более интеллектуальной, например:
%% $1,5x$ будет читаться как полтора икса, а не один запятая пять иксов.
%% Однако если написать $1, 5x$, то все будет как прежде.
\usepackage{icomma}

%% Один из пакетов, позволяющий делать таблицы на всю ширину текста.
\usepackage{tabularx}
\usepackage{subfiles}

%% Данные пакеты необязательны к использованию в бакалаврских/магистерских
%% Они нужны для иллюстративных целей
%% Начало
\usepackage{tikz}
\usetikzlibrary{arrows}
\usepackage{filecontents}
%% Конец

%% Указываем файл с библиографией.
\addbibresource{bachelor-thesis.bib}


\begin{document}

\studygroup{M3435}
\title{Предсказание забытых для модификации файлов на основе Git репозитория}
\author{Рыкунов Николай Викторович}{Рыкунов Н.В.}
\supervisor{Сметанников Иван Борисович}{Сметанников И.Б.}{доцент, к.т.н.}{научный сотрудник университета ИТМО}
\publishyear{2020}
%% Дата выдачи задания. Можно не указывать, тогда надо будет заполнить от руки.
% \startdate{01}{сентября}{2018}
%% Срок сдачи студентом работы. Можно не указывать, тогда надо будет заполнить от руки.
% \finishdate{31}{мая}{2019}
%% Дата защиты. Можно не указывать, тогда надо будет заполнить от руки.
% \defencedate{15}{июня}{2019}

\addconsultant{Поваров Н.И.}{ООО <<ИнтеллиДжей Лабс>>{}, Аналитик}

\secretary{Павлова О.Н.}

%% Эта команда генерирует титульный лист и аннотацию.
\maketitle{Бакалавр}

%% Оглавление
\tableofcontents

%% Макрос для введения. Совместим со старым стилевиком.
\startprefacepage

\subfile{intro}

%% Начало содержательной части.
\chapter{Обзор предметной области используемой для предсказания забытых для модификации файлов на основе Git репозитория}
\subfile{chapter1}

\chapter{Использованные методы и инструменты для реализации плагина предсказания забытых для модификации файлов на основе Git репозитория}
\subfile{chapter2}

\chapter{Обучение и оценка качества моделей предсказания забытых для модификации файлов на основе Git репозитория}
\subfile{chapter3}

\chapter{Реализация плагина для предсказания забытых для модификации файлов на основе Git репозитория с использованием модели предсказаний}
\subfile{chapter4}

\startconclusionpage
\subfile{conclusion}

\printmainbibliography

\appendix

\chapter{Персонализация решающей функции}
\subfile{code/personalization}
\end{document}