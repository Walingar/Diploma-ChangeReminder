\documentclass[times]{itmo-student-thesis}

%% Опции пакета:
%% - specification - если есть, генерируется задание, иначе не генерируется
%% - annotation - если есть, генерируется аннотация, иначе не генерируется
%% - times - делает все шрифтом Times New Roman, собирается с помощью xelatex
%% - languages={...} - устанавливает перечень используемых языков. По умолчанию это {english,russian}.
%%                     Последний из языков определяет текст основного документа.

%% Делает запятую в формулах более интеллектуальной, например:
%% $1,5x$ будет читаться как полтора икса, а не один запятая пять иксов.
%% Однако если написать $1, 5x$, то все будет как прежде.
\usepackage{icomma}

%% Один из пакетов, позволяющий делать таблицы на всю ширину текста.
\usepackage{tabularx}

%% Данные пакеты необязательны к использованию в бакалаврских/магистерских
%% Они нужны для иллюстративных целей
%% Начало
\usepackage{tikz}
\usetikzlibrary{arrows}
\usepackage{filecontents}
%% Конец

%% Указываем файл с библиографией.
\addbibresource{bachelor-thesis.bib}

\begin{document}

\studygroup{M3435}
\title{Предсказание забытых для модификации файлов на основе Git репозитория}
\author{Рыкунов Николай Викторович}{Рыкунов Н.В.}
\supervisor{Сметанников Иван Борисович}{Сметанников И.Б.}{доцент, к.т.н.}{научный сотрудник университета ИТМО}
\publishyear{2020}
%% Дата выдачи задания. Можно не указывать, тогда надо будет заполнить от руки.
\startdate{01}{сентября}{2018}
%% Срок сдачи студентом работы. Можно не указывать, тогда надо будет заполнить от руки.
\finishdate{31}{мая}{2019}
%% Дата защиты. Можно не указывать, тогда надо будет заполнить от руки.
\defencedate{15}{июня}{2019}

\addconsultant{Поваров Н.И.}{ООО "ИнтеллиДжей Лабс"{}, Аналитик}

\secretary{Павлова О.Н.}

%% Эта команда генерирует титульный лист и аннотацию.
\maketitle{Бакалавр}

%% Оглавление
\tableofcontents

%% Макрос для введения. Совместим со старым стилевиком.
\startprefacepage

В данном разделе размещается введение.

%% Начало содержательной части.
\chapter{Первая глава}

%% Так помечается начало обзора.
\startrelatedwork
Пример ссылок в рамках обзора: \cite{example-english, example-russian, unrestricted-jump-evco, doerr-doerr-lambda-lambda-self-adjustment-arxiv}.
%% Так помечается конец обзора.
\finishrelatedwork
Вне обзора:~\cite{bellman}.


\chapter{Перевод статьи}
\section{Введение}
Git является распределенной системой контроля версий. Система контроля версий помогает разработчику сохранить проделанную работу локально или на сервер. Проектная информация хранится в специальной базе данных, которая называется репозиторий. Репозиторий состоит из некоторого множества снимков изменений, которые именуются коммит. Каждый коммит в свою очередь состоит из ссылки на предыдущий коммит (если имеется), мета-информации (сообщение, автор, время коммита), и самих зафиксированных изменений. В среднем разработчик в большом проекте делает несколько коммитов в день. Они фиксируются в репозитории. Проектов, использующих систему контроля версий Git, большое количество. Таким образом, на данный момент накоплена очень большая кодовая база. Её часть является открытой, например, всем известный сайт GitHub на январь 2020 года содержал более чем 28 миллионов открытых репозиториев. Такая кодовая база неявно хранит в себе шаблоны поведения разработчиков, настройки их окружения и другие интересные для анализа данные. Используя этот набор данных, мы можем помочь разработчикам избежать ошибок, приводивших к неконсистентному состоянию репозитория в прошлом.
\section{Работы}

\section{Постановка задачи}
\section{Датасет}
\section{Решения}
\section{Планы на будущее}

\end{document}