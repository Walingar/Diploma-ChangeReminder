\documentclass[times,specification,annotation]{itmo-student-thesis}

\usepackage{icomma}

\usepackage{tabularx}
\usepackage{subfiles}

\usepackage{tikz}
\usetikzlibrary{arrows}
\usepackage{filecontents}
\addbibresource{bachelor-thesis.bib}

\begin{document}

\studygroup{M3435}
\title{Предсказание забытых для модификации файлов на основе Git репозитория}
\author{Рыкунов Николай Викторович}{Рыкунов Н.В.}
\supervisor{Сметанников Иван Борисович}{Сметанников И.Б.}{доцент, к.т.н.}{научный сотрудник университета ИТМО}
\publishyear{2020}

\addconsultant{Поваров Н.И.}{ООО <<ИнтеллиДжей Лабс>>{}, Руководитель команды анализа данных}

\secretary{Павлова О.Н.}

%% Задание
%%% Техническое задание и исходные данные к работе
\technicalspec{ Требуется разработать плагин для платформы IntelliJ, который во время пользовательского коммита будет рекомендовать пользователю модифицировать и добавлять в коммит файлы из репозитория. Плагин должен быть основан на модели машинного обучения. В худшем случае потребление оперативной памяти плагином должно быть не более 10 Мегабайт. }

%%% Содержание выпускной квалификационной работы (перечень подлежащих разработке вопросов)
\plannedcontents{\begin{enumerate}
    \item обзор и анализ существующих решений задачи предсказания забытых для модификации файлов на основе Git репозитория;
    \item разработка и реализация модели машинного обучения для рекомендации файлов забытых для модификации. Оценка качества обученной модели;
    \item разработка и реализация архитектуры плагина для платформы IntelliJ, основанного на обученной модели.
\end{enumerate}}

%%% Исходные материалы и пособия 
\plannedsources{\begin{enumerate}
    \item Mohri M., Rostamizadeh A., Talwalkar A. Foundations of Machine Learning.~--- The MIT Press, 2012.;
    \item Chacon S., Straub B. Pro Git.~--- USA : Apress, 2014.;
    \item Christakis M., Bird C. What developers want and need from program analysis: An empirical study ~--- 31st IEEE/ACM International Conference on Automated Software Engineering, 2016.
\end{enumerate}}

%%% Цель исследования
\researchaim{реализовать плагин для платформы IntelliJ, предсказывающий забытые для модификации файлы на основе Git репозитория}

%%% Задачи, решаемые в ВКР
\researchtargets{\begin{enumerate}
    \item разработка и обучение моделей машинного обучения для рекомендации файлов забытых для модификации на основе Git репозитория. Оценка качества обученных моделей и сравнение между собой;
    \item разработка и реализация архитектуры двух плагинов, предсказывающих забытые для модификации файлы на основе Git репозитория, для платформы IntelliJ;
    \item оценка качества моделей, используемых в плагинах для платформы IntelliJ, исходя из собранной статистики.
\end{enumerate}}

%%% Использование современных пакетов компьютерных программ и технологий
\addadvancedsoftware{Интегрированная среда разработки IntelliJ IDEA}{\ref{chapter3}, \ref{chapter4}}
\addadvancedsoftware{Язык программирования Kotlin}{\ref{chapter3}, \ref{chapter4}}
\addadvancedsoftware{Язык программирования Python}{\ref{chapter3}}

%%% Краткая характеристика полученных результатов 
\researchsummary{ Было разработано несколько моделей предсказания забытых для модификации файлов на основе Git репозитория. С использованием некоторых из них было реализовано два плагина: GitAlso и ChangeReminder. Плагин GitAlso был добавлен в репозиторий плагинов. Плагин ChangeReminder был добавлен по умолчанию в IDE на платформе IntelliJ начиная с версии 2019.2. }

%%% Гранты, полученные при выполнении работы 
\researchfunding{ При выполнении работы грантов получено не было. }

%%% Наличие публикаций и выступлений на конференциях по теме выпускной работы
\researchpublications{ При выполнении работы публикаций и выступлений не было. }

%% Эта команда генерирует титульный лист и аннотацию.
\maketitle{Бакалавр}

%% Оглавление
\tableofcontents

\startrelatedwork
%% Макрос для введения. Совместим со старым стилевиком.
\startprefacepage

\subfile{intro}

%% Начало содержательной части.
\chapter{Обзор предметной области используемой для предсказания забытых для модификации файлов на основе Git репозитория}
\subfile{chapter1}
\finishrelatedwork
\chapter{Использованные методы и инструменты для реализации плагина предсказания забытых для модификации файлов на основе Git репозитория}
\subfile{chapter2}

\chapter{Обучение и оценка качества моделей предсказания забытых для модификации файлов на основе Git репозитория}\label{chapter3}
\subfile{chapter3}

\chapter{Реализация плагина для предсказания забытых для модификации файлов на основе Git репозитория с использованием модели предсказаний}\label{chapter4}
\subfile{chapter4}

\startconclusionpage
\subfile{conclusion}

\printmainbibliography

\appendix

\chapter{Персонализация решающей функции}
\subfile{code/personalization}
\end{document}