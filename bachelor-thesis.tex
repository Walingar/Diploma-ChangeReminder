\documentclass[times,specification,annotation]{itmo-student-thesis}

%% Опции пакета:
%% - specification - если есть, генерируется задание, иначе не генерируется
%% - annotation - если есть, генерируется аннотация, иначе не генерируется
%% - times - делает все шрифтом Times New Roman, собирается с помощью xelatex
%% - languages={...} - устанавливает перечень используемых языков. По умолчанию это {english,russian}.
%%                     Последний из языков определяет текст основного документа.

%% Делает запятую в формулах более интеллектуальной, например:
%% $1,5x$ будет читаться как полтора икса, а не один запятая пять иксов.
%% Однако если написать $1, 5x$, то все будет как прежде.
\usepackage{icomma}

%% Один из пакетов, позволяющий делать таблицы на всю ширину текста.
\usepackage{tabularx}
\usepackage{subfiles}

%% Данные пакеты необязательны к использованию в бакалаврских/магистерских
%% Они нужны для иллюстративных целей
%% Начало
\usepackage{tikz}
\usetikzlibrary{arrows}
\usepackage{filecontents}
%% Конец

%% Указываем файл с библиографией.
\addbibresource{bachelor-thesis.bib}


\begin{document}

\studygroup{M3435}
\title{Предсказание забытых для модификации файлов на основе Git репозитория}
\author{Рыкунов Николай Викторович}{Рыкунов Н.В.}
\supervisor{Сметанников Иван Борисович}{Сметанников И.Б.}{доцент, к.т.н.}{научный сотрудник университета ИТМО}
\publishyear{2020}
%% Дата выдачи задания. Можно не указывать, тогда надо будет заполнить от руки.
% \startdate{01}{сентября}{2018}
%% Срок сдачи студентом работы. Можно не указывать, тогда надо будет заполнить от руки.
% \finishdate{31}{мая}{2019}
%% Дата защиты. Можно не указывать, тогда надо будет заполнить от руки.
% \defencedate{15}{июня}{2019}

\addconsultant{Поваров Н.И.}{ООО <<ИнтеллиДжей Лабс>>{}, Аналитик}

\secretary{Павлова О.Н.}

%% Задание
%%% Техническое задание и исходные данные к работе
\technicalspec{Требуется разработать стилевой файл для системы \LaTeX, позволяющий оформлять бакалаврские работы и магистерские диссертации
на кафедре компьютерных технологий Университета ИТМО. Стилевой файл должен генерировать титульную страницу пояснительной записки,
задание, аннотацию и содержательную часть пояснительной записк. Первые три документа должны максимально близко соответствовать шаблонам документов,
принятым в настоящий момент на кафедре, в то время как содержательная часть должна максимально близко соответствовать ГОСТ~7.0.11-2011
на диссертацию.}

%%% Содержание выпускной квалификационной работы (перечень подлежащих разработке вопросов)
\plannedcontents{Пояснительная записка должна демонстрировать использование наиболее типичных конструкций, возникающих при составлении
пояснительной записки (перечисления, рисунки, таблицы, листинги, псевдокод), при этом должна быть составлена так, что демонстрируется
корректность работы стилевого файла. В частности, записка должна содержать не менее двух приложений (для демонстрации нумерации рисунков и таблиц
по приложениям согласно ГОСТ) и не менее десяти элементов нумерованного перечисления первого уровня вложенности (для демонстрации корректности
используемого при нумерации набора русских букв).}

%%% Исходные материалы и пособия 
\plannedsources{\begin{enumerate}
    \item ГОСТ~7.0.11-2011 <<Диссертация и автореферат диссертации>>;
    \item С.М. Львовский. Набор и верстка в системе \LaTeX;
    \item предыдущий комплект стилевых файлов, использовавшийся на кафедре компьютерных технологий.
\end{enumerate}}

%%% Цель исследования
\researchaim{Разработка удобного стилевого файла \LaTeX
             для бакалавров и магистров кафедры компьютерных технологий.}

%%% Задачи, решаемые в ВКР
\researchtargets{\begin{enumerate}
    \item обеспечение соответствия титульной страницы, задания и аннотации шаблонам, принятым в настоящее время на кафедре;
    \item обеспечение соответствия содержательной части пояснительной записки требованиям ГОСТ~7.0.11-2011 <<Диссертация и автореферат диссертации>>;
    \item обеспечение относительного удобства в использовании~--- указание данных об авторе и научном руководителе один раз и в одном месте, автоматический подсчет числа тех или иных источников.
\end{enumerate}}

%%% Использование современных пакетов компьютерных программ и технологий
\addadvancedsoftware{Пакет \texttt{tabularx} для чуть более продвинутых таблиц}{}
\addadvancedsoftware{Пакет \texttt{biblatex} и программное средство \texttt{biber}}{Список использованных источников}

%%% Краткая характеристика полученных результатов 
\researchsummary{Получился, надо сказать, практически неплохой стилевик. В 2015--2018 годах
его уже использовали некоторые бакалавры и магистры. Надеюсь на продолжение.}

%%% Гранты, полученные при выполнении работы 
\researchfunding{Автор разрабатывал этот стилевик исключительно за свой счет и на
добровольных началах. Однако значительная его часть была бы невозможна, если бы
автор не написал в свое время кандидатскую диссертацию в \LaTeX,
а также не отвечал за формирование кучи научно-технических отчетов по гранту,
известному как <<5-в-100>>, что происходило при государственной финансовой поддержке
ведущих университетов Российской Федерации (субсидия 074-U01).}

%%% Наличие публикаций и выступлений на конференциях по теме выпускной работы
\researchpublications{По теме этой работы я (к счастью!) ничего не публиковал.}

%% Эта команда генерирует титульный лист и аннотацию.
\maketitle{Бакалавр}

%% Оглавление
\tableofcontents

\startrelatedwork
%% Макрос для введения. Совместим со старым стилевиком.
\startprefacepage

\subfile{intro}

%% Начало содержательной части.
\chapter{Обзор предметной области используемой для предсказания забытых для модификации файлов на основе Git репозитория}
\subfile{chapter1}
\finishrelatedwork
\chapter{Использованные методы и инструменты для реализации плагина предсказания забытых для модификации файлов на основе Git репозитория}
\subfile{chapter2}

\chapter{Обучение и оценка качества моделей предсказания забытых для модификации файлов на основе Git репозитория}
\subfile{chapter3}

\chapter{Реализация плагина для предсказания забытых для модификации файлов на основе Git репозитория с использованием модели предсказаний}
\subfile{chapter4}

\startconclusionpage
\subfile{conclusion}

\printmainbibliography

\appendix

\chapter{Персонализация решающей функции}
\subfile{code/personalization}
\end{document}