\section{Машинное обучение}
    \subsection{Обучение с учителем}
    \subsection{Перекрестная проверка}
    \subsection{Задача классификации}
    \subsection{Деревья решений}
    \subsection{A/B эксперимент}
\section{Системы контроля версий}
    \subsection{Предпосылки и причины появления систем контроля версий}
    \subsection{Обзор существующих систем контроля версий}
    \subsection{Система контроля версий Git}
\section{Интегрированные среды разработки}
    \subsection{Предпосылки и причины появления интегрированных сред разработки}
    \subsection{Обзор существующих интегрированных сред разработки}
    \subsection{Платформа для создания интегрированных сред разработки IntelliJ}


\section{Обзор материалов посвященных анализу Git репозиториев и поиску зависимостей в них}
Перед началом выполнения выпускной работы были проанализированы несколько статей на тему дипломной работы и смежные темы. В мировом индексе представлено множество работ, где рассказано о том, как можно использовать исторические данные в процессе разработки программного обеспечения с целью повышения качества этого процесса. Был подход использования кодовой базы для поиска логически связанных модулей, чтобы отследить, какие файлы изменяются вместе \cite{logical-modules}. В статье \cite{source-change} автор разрабатывал систему предсказаний о том, какие изменения внесет разработчик в кодовую базу. Были проанализированы статьи, которые не имеют непосредственного отношения к выпускной работе, но подходы из этих статей использовались при разработке модели предсказаний. Например, одна из таких статей \cite{project-memory}. В данной статье авторы разработали приложение <<Hipikat>>, позволяющее новым участникам быстрее ознакомится с программной системой, с которой им предстоит работать.


\section{Формальная постановка задачи предсказания забытых для модификации файлов на основе Git репозитория}
Введем понятие репозиторий (от английского repository)~-- это ориентированный ациклический граф, где каждое ребро представляет из себя отношение <<быть родителем>>. Вершину данного графа будем называть коммит (от английского commit). Коммит содержит в себе свойства: множество файлов $Files \subseteq File^k$ и мета-информацию (время создания, автор, название коммита). $File \in \mathbb{N}$ -- идентификатор файла в репозитории.\\
\subsection{Требования к модели рекомендации}\label{ml-model-req}
Пусть нам даны репозиторий и коммит из него. Файлы из этого коммита обозначим $TargetCommitFiles \subseteq File^k$. Требуется по данным множеству $CommitFiles \subseteq TargetCommitFiles$, мета-информации о коммите и всех потомках коммита возвращать $TargetCommitFiles$.
\subsection{Требования к реализации плагина для IntelliJ платформы}\label{impl-req}
Требуется разработать плагин для платформы IntelliJ, который во время пользовательского коммита будет рекомендовать пользователю модифицировать и добавлять в коммит файлы из репозитория. Плагин должен быть основан на модели, требования к которой изложены в \ref{ml-model-req}. В худшем случае время работы плагина должно составлять не более двух секунд на процессоре 2,2 GHz Quad-Core Intel Core i7. В худшем случае, потребление памяти должно быть не более 10 Мегабайт.

\chapterconclusion