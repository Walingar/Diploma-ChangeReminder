\section{Методы и инструменты использованные для разработки и реализации модели рекомендации}
    \subsection{Ассоциативные правила}
    \subsection{Байесовское среднее}
    \subsection{Случайный лес}
\section{Методы и инструменты использованные для разработки и реализации IntelliJ плагина}
    \subsection{IntelliJ платформа}
    \subsection{VCS часть платформы}
    \subsection{Плагин <<Git4Idea>>}
\section{Существующие решения задачи поиска релевантных файлов в Git репозитории}
На данный момент существует решение поставленной задачи, которое называется <<git-also>>. Программа является приложением для командной строки. Оно позволяет для заданного файла получить набор файлов, с которыми введенный файл был в большем числе коммитов вместе. Данное решение не соответствует требованию, представленному в пункте \ref{impl-req}, т.к. приложение не является плагином для платформы <<IntelliJ>>. Также, данное приложение может решать задачу поставленную в пункте \ref{ml-model-req} только с использованием множества запросов. При внимательном рассмотрении данного решения, можно заметить, что оно использует простую решающую функцию, которая будет являться для нас базовым показателем (см. пункт \ref{baseline}).
\chapterconclusion