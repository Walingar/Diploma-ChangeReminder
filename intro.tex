Git является распределенной системой контроля версий. Система контроля версий помогает разработчику сохранить проделанную работу локально или на сервер. Проектная информация хранится в специальной базе данных, которая называется репозиторий. Репозиторий состоит из некоторого множества снимков изменений, которые именуются коммит. Каждый коммит в свою очередь состоит из ссылки на предыдущий коммит (если таковой имеется), мета-информации (сообщение, автор, время коммита) и самих зафиксированных изменений. В среднем разработчик в большом проекте делает несколько коммитов в день, которые фиксируются в репозитории. Количество проектов, использующих систему контроля версий Git, существует огромное множество, таким образом образуя большую кодовую базу. Некоторая её часть является открытой, например, сайт GitHub на январь 2020 года содержал более чем 28 миллионов открытых репозиториев. Такая кодовая база неявно хранит в себе шаблоны поведения разработчиков, настройки их окружения и другие интересные для анализа данные. Используя этот набор данных, мы можем помочь разработчикам избежать ошибок, приводивших к неконсистентному состоянию репозитория в прошлом.